%        File: test.tex
%     Created: Thu Apr 21 03:00 PM 2011 C
% Last Change: Thu Apr 21 03:00 PM 2011 C
%

\documentclass[a4paper]{article}

\usepackage{amsmath}
\usepackage{graphicx}

\begin{document}

\section{Problem}
We have sampled a set of $m$ individuals $I$ from the general population. For each individual $i \in I$ we have gathered some information:
\begin{itemize}
  \item A matrix $V$ of size $m \times l$ that contains the set of sampled SNPs for each individual. Each element $s_{i,j} \in \{0,1,2\}$ describes the number of alleles for SNP $j$ for individual $i$.
  \item A matrix $R$ of risk factors of size $n \times k$. Each element $r_{i,j} \in \{0,1\}$ describes the risk factor $j$ for individual $i$.
  \item A vector $c$ of length $n$ that classifies each individual $i$ as healthy ($c_i = 0$) or sick ($c_i = 1$).
\end{itemize}
The problem is to find risk factors that are good predictors of the disease.

\section{Method}
The core idea is to find a set of axis aligned hyperplanes that maximizes the area under the resulting ROC curve.
\begin{itemize}
  \item Values $x_{i,l}$ (e.g. an individual SNP and risk factors) where $i \in [m]$ and $l \in [k]$.
  \item Classification $c_i$ where $i \in [m]$.
  \item Let $m^+$ be the number of postive samples and $m^-$ be the number of negative samples.
\end{itemize}
We first realize that the Area Under the Curve (AUC) for the ROC-cruve can be shown to be:
\begin{equation}
AUC = Pr[h(X^+) > h(X^-)] = E[I_{\{h(X^+)\ >\ h(X^-)\}}]
\end{equation}
where $X^+$ and $X^-$ is the random variable denoting a postive and negative sample respectively, and $h$ is our hypothesis function, i.e. that predicts whether a sample is sick or healthy. The area under the ROC cruve is therefore the probability that a postive sample is predicted as a positive sample. The next thing is to note that equation (1) can be approximated by
$$E[I_{\{h(X^+)\ >\ h(X^-)\}}] \approx \frac{ \sum_{i=1}^{m^+}\sum_{j=1}^{m^-} I_{\{h(x_i^+) > h(x_j^-)\} }}{m^+ \cdot m^-}$$
by the law of large numbers. With this we can set up the positioning of the hyperplanes as a quadratic programming problem. A quadratic programming problem has the following form:
\begin{align*}
&\text{Maximize} & f(\mathbf{x}) = \mathbf{x}^TQ\mathbf{x} + \mathbf{c}^T\mathbf{x}\\
&\text{subject to} & A\mathbf{x} \leq b
\end{align*}
In our case we will have the area under the ROC curve as our $f$ and $\mathbf{x}$ will be a vector of variables denoted $n$. We define $n_{i,l}$ to mean what side of the hyperplane indvidual $i$ is in dimension $l$, $0$ means left and $1$ means right. We assume that in an optimal solution each $n_{i,l}$ will be either $0$ or $1$ (proof). The only condition is the following:
\begin{itemize}
  \item $0 \leq n_{i_1,l} \leq \dots \leq n_{i_m,l} \leq 1$ where $i_1, \dots, i_m$ is a permutation such that $x_{i_1,l} \leq \dots \leq x_{i_m,l}$. This conditions says that the values projected down to dimension $l$ will be split into two intervals, and also that each $n_{i,l}$ must be between $0$ and $1$ (hopefully either $0$ or $1$).
\end{itemize}
The part with $f(n) = AUC(n)$ is a bit tricky, we use equation (1) to formulate it as:
$$ f(n) = \sum_{i=1}^{m^+}\sum_{j=1}^{m^-} \sum_{l=1}^k n_{i,l}(1 - n_{j,l}) + (1-n_{i,l})n_{j,l} =$$
$$ = \sum_{i=1}^{m^+}\sum_{j=1}^{m^-} \sum_{l=1}^k -2n_{i,l}n_{j,l} + n_{i,l} + n_{j,l}$$
So in each dimension they are in different half planes they get one +.
\end{document}
